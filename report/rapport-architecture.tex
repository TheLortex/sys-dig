\documentclass[a4paper]{article}
\usepackage[utf8]{inputenc}
\usepackage[T1]{fontenc}
\usepackage{amsmath}
\usepackage{amssymb}
\usepackage{amsthm}
\usepackage{listings}
\usepackage{enumitem}
\usepackage{relsize}
\usepackage{dsfont}
\usepackage{array}
\usepackage[margin=1.0in]{geometry}

\title{Architecture du processeur}

\begin{document}

\maketitle

\section{Mémoires et registres}

Le procésseur possède une unique RAM et une unique ROM. Au démarrage de la
simulation, la RAM est entièrement vide. La ROM contient le programme sous forme
d'opcodes, générés par du code assembleur.

Le processeur possède 16 registres, de 32 bits.

\section{Opcodes}

0-3 : Conditions flags\\
-N -> négatif strictement
-Z -> zéro
-C -> dépasse 32 bits
-V -> dépasse 31 bits

4-5 : On sait pas encore

6 : 1 = immediat, sinon registre

7-10 : opcodes

11 : changer les flags ou pas

12-15 : Registre première opération

16-19 : Registre de destination

20-31 : operande 2


Operande 2 registre :
0-7 : shift
8-11 : registre de l'opération 2

Operande 2 immediat :
0-3 : shift
4-11 : immediat

\section{Instructions assembleur}

\subsection{Opérations arithmétiques et booléennes}

\begin{tabular}{|c|c|c|}
  \hline
  Opcode & Nom assembleur & effet \\
  \hline
  0000 & AND & R[rd] = op1 and op2\\
  0001 & EOR & R[rd] = op1 xor op2\\
  0010 & SUB & R[rd] = op1 - op2\\
  0011 & RSB & R[rd] = op2 - op1\\
  0100 & ADD & R[rd] = op1 + op2\\
  0101 & ADC & R[rd] = op1 + op2 + c\\
  0110 & SBC & R[rd] = op1 - op2 + c - 1\\
  0111 & RSC & R[rd] = op2 - op1 + c - 1\\
  1000 & TST & set les flags sur op1 and op2\\
  1001 & TEQ & set les flags sur op1 xor op2\\
  1010 & CMP & set les flags sur op1 - op2\\
  1011 & CMN & set les flags sur op1 + op2\\
  1100 & ORR & R[rd] = op1 or op2\\
  1101 & MOV & R[rd] = op2\\
  1110 & BIC & R[rd] = op1 nand op2\\
  1111 & MVN & R[rd] = not op2\\
  \hline
\end{tabular}

Où c représente le carry

\subsection{Accès à la mémoire}

\begin{tabular}{|c|c|c|}
  \hline
  Opcode & Nom assembleur & effet\\
  \hline
  0 & LDR & R[Rd] = mémoire à l'adresse op2\\
  1 & STR & mémoire à l'adresse op2 = R[Rd]\\
  \hline
            
\end{tabular}

\section{Détail de l'architecture}

\subsection{ALU}



architecture et fonctionnement de l'ALU\\
architecture et fonctionnement de la carte graphique\\
fonctionnement de la lecture d'instructions ?

\end{document}
