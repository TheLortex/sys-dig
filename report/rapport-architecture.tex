\documentclass[a4paper]{article}
\usepackage[utf8]{inputenc}
\usepackage[T1]{fontenc}
\usepackage{amsmath}
\usepackage{amssymb}
\usepackage{amsthm}
\usepackage{listings}
\usepackage{enumitem}
\usepackage{relsize}
\usepackage{dsfont}
\usepackage[margin=1.0in]{geometry}

\title{Architecture du processeur}

\begin{document}

\maketitle

\section{Mémoires et registres}

(Combien de registres, RAM, ROM)\\
fonctions spéciales des registres (program counter, registre d'entrée, de sortie ?)

\section{Opcodes}
\begin{tabular}{}
  \hline
  0-3 & 4 & 5 & 6 & 7-10 & 11 & 12-15 & 16-19 & 20-31
  \hline
  

\begin{tabular}{|c|c|}
\hline
    0 - 3 & Conditionnelles
  \hline
    4 & Réservé
  \hline
    5 & R: Si R = 1: opérations de RAM. Si R = 0: opérations arithmétiques et logiques.
  \hline
    6 & Opération immédiate I
  \hline
    7 - 10 & Opcode
  \hline
    11 & Sauvegarder les flags
  \hline
    12 - 15 & Registre op1
  \hline
    16 - 19 & Registre rd
  \hline
    20 - 31 & Opérande op2
  \hline
\end{tabular}
\subsection{Flags}
\begin{itemize}
  \item N : Négatif
  \item Z : Zéro
  \item C : Carry
  \item V : Overflow
\end{itemize}
\subsection{Conditionnelles}
\begin{tabular}{|c|c|}
  \hline
  Code & Nom & Condition sur les flags & Signification (pour CMP ou SUB)
  \hline
  0000 & EQ & Z & Égalité\\
  0001 & NEQ & \bar{Z} & Non égalité\\
  0010 & CS/HS & C & Carry set\\
  0011 & CC/LO & \bar{C} & Carry clear\\
  0100 & MI & N & Négatif \\
  0101 & PL & \bar{N} & Positif ou nul \\
  0110 & VS & V & Overflow signé \\
  0111 & VC & \bar{V} & Pas d'overflow signé \\
  1000 & HI & C \&\& \bar{Z} & Strictement plus grand non signé \\
  1001 & LS & \bar{C} \|\| Z & Plus petit non signé \\
  1010 & GE & N == V & Plus grand signé \\
  1011 & LT & N != V & Strictement plus petit signé\\
  1100 & GT & \bar{Z} \&\& (N == V) & Strictement plus grand signé \\
  1101 & LE & Z \|\| (N != V) & Plus petit signé \\
  1111 & AL &  & Toujours exécuté \\
\end{tabular}
\subsection{Registres}
16 registres addressés sur 4 bits.\\
Le registre \%r00 pointe sur la prochaine instruction à exécuter dans la ROM.
\subsection{Opérande op2}
L'opérande op2 est selon la valeur de I soit un registre soit une constante.
Si I = 1:
\begin{tabular}{|c|c|}
  \hline
  0 -


\section{Instructions assembleur}

Liste des opérations assembleur (add, etc..)\\
plus les flags

\section{Détail de l'architecture}

architecture et fonctionnement de l'ALU\\
architecture et fonctionnement de la carte graphique\\
fonctionnement de la lecture d'instructions ?

\end{document}
