\documentclass[a4paper]{article}
\usepackage[utf8]{inputenc}
\usepackage[T1]{fontenc}
\usepackage{amsmath}
\usepackage{amssymb}
\usepackage{amsthm}
\usepackage{listings}
\usepackage{enumitem}
\usepackage{relsize}
\usepackage{dsfont}
\usepackage{array}
\usepackage[margin=1.0in]{geometry}

\title{Architecture du processeur}
\author{Pluvinage Lucas, Fehr Mathieu, Dang-nhu Hector, Voss Malachi}
\begin{document}

\maketitle

\section{Organisation du projet}

Le projet est organisé en plusieurs dosseiers :
\begin{itemize}
  \item as contient un compilateur allant de la syntaxe ARM vers les
    instructions lues par notre assembleur
  \item minijazz contient une version modifiée de minijazz
  \item mjblocks contient les fontions minijazz représentant le processeur
  \item panda contient un compilateur du langage bamboo (fortement inspiré du
    basic) vers le langage ARM
  \item simulateur\_netlist contient un compilateur de netlists, pour la
    simulation
  \item clock contient des scripts pour éxecuter facilement l'horloge 
\end{itemize}

\section{Modèle de processeur et architecture globale}

Nous avons choisi de beaucoup nous inspirer des processeurs ARM, qui ont
l'avantage d'avoir un jeu d'instruction à la fois simple et complet, tout en
possédant une certaine efficacité grace aux conditionnement des instructions,
permettant de réduire le nombre de jump à effectuer dans les instructions
assembleurs.

\subsection{Mémoires et registres}

Le processeur que nous utilisons possède 16 registres, dont un program counter.
Il possède une unique ROM addressée sur 16 bits, dont les éléments sont des
entiers binaires de 32 bits. Notre processeur n'utilises pas de RAM, bien que
les registres soient modélisés sous forme de RAM adressée sous 4 bits.
La ROM contient le programme sous forme d'opcodes, générés par du code
assembleur ARM.

Le registre rF est le program counter, et pointe sur l'instruction courante.
Le registre rE est utilisé par les instructions "branch and link" qui sauvegarde
la valeur de rF avant de faire un saut.

\subsection{Opcodes}

Une instruction assembleur est écrite sous 32 bits, et est structurée ainsi :

\begin{tabular}{|c|c|c|c|c|c|c|c|c|c|c|c|}
  \hline
  0-3  & 4 & 5 & 6 & 7 & 8-10 & 11 & 12-15 & 16-19 & 20-23 & 24-27 & 28-31 \\
  \hline
  COND & 0 & 0 & I & \multicolumn{2}{c|}{OPCODE} & Set flag & Registre rd & Registre op1 & \multicolumn{3}{c|}{Opérande op2}  \\
  \hline
  COND & 0 & 0 & 1 & \multicolumn{2}{c|}{OPCODE} & Set flag & Registre rd &Registre op1 & Shift & \multicolumn{2}{c|}{Constante} \\
  \hline
  COND & 0 & 0 & 0 & \multicolumn{2}{c|}{OPCODE} & Set flag & Registre rd & Registre op1 & \multicolumn{2}{c|}{Shift} & r2  \\
  \hline
  COND & 1 & 0 & 1 & linking & \multicolumn{ 7}{c|}{Offset} \\
  \hline
  COND & 1 & 1 & 1 & 0 & \multicolumn{4}{c|}{Port} & \multicolumn{3}{c|}{Valeur} \\
  \hline
\end{tabular}

\subsubsection{Flags}

A chaque opération, si le booléen set flag est activé ou que l'opération est une opération sans registre de destination, alors les flags sont
recalculés :
  \begin{itemize}
  \item N : La dernière instruction renvoie un entier négatif
  \item Z : La dernière instruction renvoie zéro
  \item C : La dernière instruction fait un dépassement dans la représentation
    non signée
  \item V : La dernière instruction fait un dépassement dans la représentation
    signée
  \end{itemize}

  \subsubsection{Conditionnelles}

  Voici les codes des conditions d'executions des opérations:\\

\begin{tabular}{|c|c|c|c|}
  \hline
  Code & Nom & Condition sur les flags & Signification (pour CMP ou SUB)\\
  \hline
  0000 & EQ & Z & Égalité\\
  0001 & NEQ & $\bar{Z}$ & Non égalité\\
  0010 & CS/HS & C & Carry set\\
  0011 & CC/LO & $\bar{C}$ & Carry clear\\
  0100 & MI & N & Négatif \\
  0101 & PL & $\bar{N}$ & Positif ou nul \\
  0110 & VS & V & Overflow signé \\
  0111 & VC & $\bar{V}$ & Pas d'overflow signé \\
  1000 & HI & C and $\bar{Z}$ & Strictement plus grand non signé \\
  1001 & LS & $\bar{C}$ or Z & Plus petit non signé \\
  1010 & GE & N == V & Plus grand signé \\
  1011 & LT & N != V & Strictement plus petit signé\\
  1100 & GT & $\bar{Z}$ and (N == V) & Strictement plus grand signé \\
  1101 & LE & Z or (N != V) & Plus petit signé \\
  1111 & AL &  & Toujours exécuté \\
  \hline
\end{tabular}

\section{Instructions assembleur}

La syntaxe pour effectuer les opérations en assembleur suit la syntaxe spécifiée
sur les processeurs ARM. Nous avons créer un compilateur qui permet de compiler
un programme ARM vers nos instructions sur 32 bits. Le programme est sité dans
le dossier am.

\subsection{Opérations arithmétiques et booléennes}

Voici les opérations assembleur arithmétiques et booléennes :\\

\begin{tabular}{|c|c|c|}
  \hline
  Opcode & Nom assembleur & effet \\
  \hline
  0000 & ADD & R[rd] = op1 + op2\\
  0001 & ADC & R[rd] = op1 + op2 + c\\
  0010 & RSB & R[rd] = op2 - op1\\
  0011 & RSC & R[rd] = op2 - op1 + c - 1\\
  0100 & CMP & set les flags sur op1 - op2\\
  0101 & CMN & set les flags sur op1 + op2\\
  0110 & SUB & R[rd] = op1 - op2 \\
  0111 & SBC & R[rd] = op1 - op2 + c - 1\\
  1000 & AND & op1 and op2\\
  1001 & TST & set les flags sur op1 and op2\\
  1010 & ORR & op1 or op2\\
  1011 & BIC & op1 nand op2\\
  1100 & NOT & not op2\\
  1101 & MOV & op2\\
  1110 & EOR & op1 xor op2\\
  1111 & TEQ & set les flags sur op1 xor op2\\
  \hline
\end{tabular}

Où c représente le carry

\subsection{Instruction de sauts}
Deux instructions sont disponibles: branch (B) et branch and link (BL).
Branch s'occupe simplement de positionner le program counter (rF) à un autre
endroit du code. Branch and link sauvegarde en plus la valeur du program counter
incrémenté (rF+1) dans rE. Cela permet de faire des retours très facilement en
faisant un MOV rF, rE.

\subsection{Instruction spécifique au processeur}

L'instruction WAI a été ajoutée. L'opération prends deux paramètres, un port et
une valeur sur 12 bits. Le program counter reste bloqué sur l'instruction tant
que le port n'envoie pas la valeur indiquée.

Cette instruction à été ajouté pour permettre une synchronisation avec un
élément externe au processeur, comme une horloge.

\subsection{Barrel shifter}

Le barrel shifter permet d'effectuer les opérations LSL, ASL, LSR, ASR, ROR et
RRX sur l'opérande 2. Il permet aussi de faire un move avec un entier supérieur
à 8 bits.

\begin{tabular}{|c|c|c|}
  \hline
  Opcode & Nom assembleur & effet \\
  \hline
  00 & LSL/ASL & effectue un shift gauche\\
  01 & ROR & effectue une rotation droite\\
  01 & RRX & effectue une rotation droite avec le bit de carry\\
  10 & LSR & effectue un shift droit logique\\
  11 & ASR & effectue un shift gauche arithmétique\\
  \hline
\end{tabular}

\subsection{ALU}
L'ALU prend en entrée deux valeurs sur 32 bits op1 et op2, le carry sortant de la dernière instruction et l'opcode de l'instruction actuelle.
On a réorganisé les opcodes afin d'optimiser les calculs de l'ALU, pour n'avoir qu'un seul adder.

Ainsi les 8 premiers opcode (0XXX) correspondent à des instructions utilisant l'adder.
Pour transformer l'adder en soustracteur il suffit par exemple de calculer l'opposé de op1, ce qui se fait en calcul incr(not(op1)) où incr désigne l'incrémenteur.
Ainsi on arrive avec un jeu de mux à rediriger les bonnes valeurs dans l'ALU.
Les flags sortant correspondent soit au flag de l'adder, soit au flag calculé en sortie des opérations logiques. Dans le cas des opérations logiques, les flags C et V
sont toujours à 0, et seuls les flags Z et N sont calculés.
\section{Utilisation du processeur pour créer une montre}

Deux programmes assembleurs ont été écrits pour la montre : un programme
simulant une montre en temps réel, et un autre le simulant en temps accéléré
(aussi vite que le processeur le permet).

Les deux programmes ne diffèrent que de peu. La seule différence est que l'une
attends un signal donné par une horloge pour changer de seconde.

\subsection{Programme assembleur}


Le programme est structuré en plusieurs blocs assembleurs, s'appellant pas
système de branching. Il y a un bloc pour les unités de minutes, les dizaines de
minutes, les unités d'heures... La seule exception étant pour les secondes, qui
sont regroupés en un bloc. Chaque bloc s'occupe d'appeler les blocs d'unités
plus faibles autant de fois qu'il le faut. De cette manière, il n'est pas
nécessaire de garder en mémoire la valeur des unités.

La date (hormi l'année) et l'heure est stocké dans le registre r0. L'année est
elle stockée dans le registre r1. Il faut noter qu'on ne stocke pas les valeurs
en elles même, mais les valeurs des digits de chaque chiffre en décimal. Cela
permet de relier directement les 2 registres à des afficheurs 7 segments, pour
les afficher en temps réel au moment de leur modification.

Pour éviter d'avoir des heures impossibles affichées comme 3h60, notre
représentation permet de modifier toutes les valeurs (sauf les années) en une
unique instruction. Cependant, pendant quelques cycles, il sera affiché une
année antérieure lors du passage à la nouvelle année.\\


Pour démarer à la date courante, nous créons via le script getdate.py une ram
qui représente l'état de la montre à la date donnée. En spécifiant la ram lors
de la simulation du procésseur, on démarre donc à la date courante.


\subsection{Affichage en 7 segment}

Les registres r0 et r1 étant relié dans le circuit à des convertisseurs 7
segments, nous avons décidé de les afficher en utilisant la SDL2, qui tourne
dans un thread en parallèle au simulateur.

\end{document}
